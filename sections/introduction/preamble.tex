Let $F_n$ denote the free group of rank  $n$, generated by  $A \coloneq \Set{a_1,\ldots,a_n} $.
Let $U_n$ denote the free product of $n$ copies of  $\Z /2\Z$.
This should be thought of as the \emph{universal Coxeter group of rank  $n$}, since all rank  $n$ Coxeter groups are naturally a quotient of  $U_n$.
Let $S \coloneq \Set{s_1,\ldots,s_n} $ be the natural generating set for $U$, such that each  $s_i$ generates the  $i^\text{th}$ free factor.
For the whole of this document,  $n$ will be constant and arbitrary, so we will drop  $n$ from the above two notations, preferring just  $F$ and $U$.

A reflection in $F$ is a conjugate of any $f_i$ in $F$.
Denote the set of all reflections in $F$ by $R$.
Similarly, a reflection in $U$ is a conjugate of  any  $s_i$ in $U$.
Denote the set of all reflections in $U$ by  $T$.
So, a reflection in $U$ is some word $us_i u^{-1}$, where $u=u_1\ldots u_k$ is a word in $S$.
We may assume $us_iu^{-1}$ is a reduced word.
Thus, since every generator in $S$ is of order 2, we may assume that $u$ is a positive word where each factor has exponent  $+1$,  i.e.~each $u_i \in S$, and each $u_i$ is not equal to  $u_{i+1}$.

Since we will be using this notation a lot, let $s_i: w$ denote  $ws_iw^{-1}$.
Let $\phi \colon F \to U$ be the surjective homomorphism defined by $a_i \mapsto s_i$.
A lift of a reflection $s_i : u$ in $U$ is a choice of a positive or negative power for each $u_i$ in $u$, i.e.~a choice of a length $2k +1$ element in $\phi^{-1}(s_i : u) \cap R$.

Let $\C_n$ denote $\C \setminus \Set{z_1,\ldots,z_n}$.
Our free group $F$ is $\pi_1(\C_n, z_0)$ for any  $z_0 \notin \Set{z_1,\ldots,z_n} $.
We make the choice of $\Set{z_1,\ldots,z_n}$ to be $n$ evenly spaced points on the imaginary axis which are within the unit circle, and $z_0 = -1$.
We draw a line $\lambda_i$ from each  $z_i$ to $z_0$.
For reasons that will become clear, we draw these lines in a roundabout way, but then for diagramatic readability, we draw that as on the left diagram below.
\[
	\includetikz[baseline=(text57948.center)]{figs/points_on_C_simplified.tex}
	\hspace{-1em}\text{equivalent to}\quad
	\includetikz[baseline=(text57948.center)]{figs/points_on_C.tex}
	.\]
In the following, let each $\epsilon_i$ be in $\Set{\pm 1}$.
A loop $l$ in  $\pi_1(\C_n, z_0)$ represents the element $v = f^{\epsilon_1}_{\alpha_1}\cdots f^{\epsilon_k}_{\alpha_k} \in F$, if $l$ passes through each  $\lambda_i$ in the order specified in  $u$, passing from bottom to top if  $\epsilon_i=1$ and top to bottom if  $\epsilon_i=-1$.
We say that some $v \in F$ is non-crossing if there exists a non-crossing loop which represents the free reduction of $v$.
One can see pictorially that any non-crossing $u = f^{\epsilon_1}_{\alpha_1}\cdots f^{\epsilon_k}_{\alpha_k} \in F$ must be \emph{square free}.
So all $f_{\alpha_i} \neq f_{\alpha_{i+1}}$ and all  $\epsilon_k \in \Set{\pm 1} $ in this setting.

Let $l_i$ represent a loop just around $z_i$ that remains a close distance around  $z_i$ for its entirety.
\begin{lemma}
	Let $v \in F$, and let $f_i \in A$, which is the generating set for $F$.
	Any non-crossing loop representing $f_i : v$ is homotopy equivalent to $\gamma l_i \gamma^{-1}$, where gamma is a non-crossing representative of $v\Omega$ concatenated with a path close to $z_i$
\end{lemma}

We will want to talk about specific paths in $\C_n$, not just loops up to homotopy.
For this, we use the following definition,
\begin{definition}
	\label{def:free_path}
	Let $\Omega \notin \Set{z_0,\ldots,z_n}$ denote an \emph{end symbol}.
	Let $v$ be some element in $F$.
	A path representing $v\Omega$ is a path (not a loop) that starts at  $z_0$, crosses each $\lambda_i$ in the correct order and orientation as above, but then does not continue back to  $z_0$.
	Once the path has crossed the final $\lambda_k$ (in the correct orientation), it stops a short distance away from that crossing.
\end{definition}
We say that $v\epsilon$ is non-crossing if there exists a non-crossing path that represents  $u\Omega$.
For example, the following is a non-crossing path that represents $f_3^{-1}f_2f_3\Omega$.
\[
	\includetikz[]{figs/non_crossing_path.tex}
\]
A path (possibly a path or a loop) $p$ in $\C_n$ intersects the  $\lambda_i$ at certain points.
We can order these points by closeness to  $z_n$.
We say that a crossing of $p$ across  $\lambda_i$ is \emph{close}, if that crossing is the closest to  $z_i$ along all of  $p$.

\begin{lemma}
	\label{lem:reflection_close_turns}
	Let $v \in F$, and let $f_i \in A$, which is the generating set for $F$.
	The following are equivalent.
	\begin{enumerate}
		\item $f_i : v$ is non-crossing.
		\item There exists a non-crossing path representing $vf_i\Omega$ such that the final crossing of $\lambda_i$ is close.
		\item There exists a non-crossing path for $vf_i \Omega$ such that we can alter the path just before and after the central crossing of $\lambda_i$, so that this altered path is a non-crossing representative for $vf_i^{-1} \Omega$.
	\end{enumerate}
\end{lemma}
\begin{proof}
	We first show $(2) \implies (1)$.
	To construct a non-crossing path for $f_i : v$, we follow the path given by (2), then turn in a positive orientation around $z_i$, then follow backwards along our original path, keeping it close on our left as we go back to  $z_0$.
	We follow this procedure in the following example.
	\[
		\includetikz[baseline=(current bounding box.center)]{figs/non_crossing_path_example_start.tex}
		\quad\leadsto\quad
		\includetikz[baseline=(current bounding box.center)]{figs/non_crossing_path_example_end.tex}
	\]
	% In order to show $(1) \implies (2)$, assume we have a non-crossing representative loop $l$ for $f_i : v$, and that the important crossing of $\lambda_i$ is not close.
	We now show $(1) \implies (2)$.
	Let $v = f^{\epsilon_1}_{\alpha_1}f^{\epsilon_2}_{\alpha_2}\cdots f^{\epsilon_n}_{\alpha_k}$, which we may assume is freely reduced.
	Let $l$ be a non-crossing loop representing  $f_i : v$.
	We work from the middle outwards to construct a picture of what $l$ could look like.

	Before turning around $z_i$,  $l$ passed through $\lambda_{\alpha_k}$ and right after turning around  $z_i$, it passed through  $\lambda_{\alpha_k}$ in the opposite direction.
	Similarly, it passed through  $\lambda_{\alpha_{k-1}}$ just before passing through  $\lambda_{\alpha_k}$, and just after in the opposite direction, and so on.
	This gives us the following picture, where $l$ is the line in black.
	\[
		\includetikz[baseline=(current bounding box.center)]{figs/central_turn.tex}
	\]
	This picture is misleading in many ways.
	For instance, there is no reason to assume that $\lambda_x \neq \lambda_y$ for $x \neq y$.
	We have also neglected to show whether each $z_{\alpha_j}$ is above or below each  $\lambda_{\alpha_j}$.
	This is because we do not know any of the values of $\epsilon_j$.
	However, the important features of this picture are the regions, denoted $r_0, \ldots, r_k$ which are bounded by $l$ and the $\lambda$ lines.
	By working from the centre of $f_i : v$, we can see that these regions exist exactly as they are depicted in the above picture.

	Now, assume that somewhere in $l$, there is a turn closer to $z_i$, which occurs inside region $r_k$.
	To reach region $r_k$, we must go via the regions $r_{k-1}$, $r_{k-2}$ and so on.
	Suppose we turn back inside at index $m$, as depicted in the following picture.

	\[
		\includetikz[baseline=(current bounding box.center)]{figs/central_turn_back.tex}
	\]
	This would involve both ends of $l$ going back in to $r_{m}$, having just passed out of region $r_{m}$.
	The only way this could happen is if there was a subword $f_{\alpha_m}f_{\alpha_m}^{-1}$  or $f_{\alpha_m}^{-1}f_{\alpha_m}$ somewhere in $l$.
	We know this is not the case because  $v$ is a reduced word.

	We now show $(2) \implies (3)$.
	Suppose a non-crossing path $p$ representing  $vf_i\Omega$ such that the final crossing of $\lambda_i$ is close.
	Since the crossing is close, we can alter the last part of the path path by moving the crossing point arbitrarily close to $z_i$.
	We then move the path so that it goes to the left of $z_i$ and passes back through  $\lambda_i$ from top to bottom.

	We finally show  $(3) \implies (2)$.
	Suppose that all non-crossing paths representing $vf_i \Omega$ have a non-close final crossing of $\lambda_i$.
	Then at their ends, all non-crossing paths representing  $vf_i \Omega$ look like the following picture, where $\Omega$ signifies the end of the path.
	\[
		\includetikz[baseline=(current bounding box.center)]{figs/cant_move_around_point.tex}
	\]
	Call the path $p$.
	The lines $\lambda_a$ and $\lambda_b$ are the lines that occur adjacent to the closer crossing of  $\lambda_i$.
	Let  $p_\text{close}$ denote the segment of $p$ that is between  $\lambda_a$ and  $\lambda_b$.
	It is possible that $\lambda_a = \lambda_b$, but this case does not cause issues for our argument.
	Irrespective of the orientation of the crossing of $\lambda_a$ and $\lambda_b$ by $p$, the end of $p$ is inside a region bounded by $p_\text{close}$, $\labda_a$, and  $\lambda_b$.
	This region does not contain $z_i$.
	We see that there is no way to alter the end of the path so that we cross $\lambda_i$ from top to bottom at the end of the path, unless we cross $\lambda_a$ or $\lambda_b$ first.
\end{proof}
\begin{lemma}
	Let $v \in F$ and let  $f_i \in A$.
	The following are equivalent.
	\begin{enumerate}
		\item
	\end{enumerate}
\end{lemma}
\begin{theorem}
	Let $u \in U$ and $s_i \in S$.
	If there is a non-crossing lift $f_i : v$ for $s_i : u$, then it is unique.
\end{theorem}
\begin{proof}
	We do this by inducting on the length of $u$, which we denote $\ell(u)$.
	The statement is vacuous when $\ell(u) = 0$.
	Assume this is true for all $s_i$ and $\ell(u_0) < k$.
	Now suppose we have some reflection $s_i : u$ with $u = s_{\alpha_1}\cdots s_{\alpha_k}$.
	And suppose this reflection has a non-crossing lift.
	There exists a maximal strict prefix $u^\prime = s_{\alpha_1}s_{\alpha_2}\cdots s_{\alpha_j}$ of  $u$ such that  $s_{\alpha_j} : s_{\alpha_1}\cdots s_{\alpha_{j-1}}$ has a non-crossing lift.
	By induction and by \cref{lem:reflection_close_turns}, there is a unique non-crossing lift for $u_1\cdots u_j \Omega$.
	Thus, to complete our proof, we need to show that the choices of travelling positively or negatively around $u_{j},\ldots,u_k$ are unique.
	We have a choice of how to travel around
\end{proof}


