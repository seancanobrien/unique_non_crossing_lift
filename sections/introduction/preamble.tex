Let $F_n$ denote the free group of rank  $n$, generated by  $A \coloneq \Set{a_1,\ldots,a_n} $.
Let $U_n$ denote the free product of $n$ copies of  $\Z /2\Z$.
This should be thought of as the \emph{universal Coxeter group of rank  $n$}, since all rank  $n$ Coxeter groups are naturally a quotient of  $U_n$.
Let $S \coloneq \Set{s_1,\ldots,s_n} $ be the natural generating set for $U$, such that each  $s_i$ generates the  $i^\text{th}$ free factor.
For the whole of this document,  $n$ will be constant and arbitrary, so we will drop  $n$ from the above two notations, preferring just  $F$ and $U$.

A reflection in $F$ is a conjugate of any $f_i$ in $F$.
Denote the set of all reflections in $F$ by $R$.
Similarly, a reflection in $U$ is a conjugate of  any  $s_i$ in $U$.
Denote the set of all reflections in $U$ by  $T$.
So, a reflection in $U$ is some word $us_i u^{-1}$, where $u=u_1\ldots u_k$ is a word in $S$.
We may assume $us_iu^{-1}$ is a reduced word.
Thus, since every generator in $S$ is of order 2, we may assume that $u$ is a positive word where each factor has exponent  $+1$,  i.e.~each $u_i \in S$, and each $u_i$ is not equal to  $u_{i+1}$.

Since we will be using this notation a lot, let $s_i: w$ denote  $ws_iw^{-1}$.
Let $\phi \colon F \to U$ be the surjective homomorphism defined by $a_i \mapsto s_i$.
A lift of a reflection $s_i : u$ in $U$ is a choice of a positive or negative power for each $u_i$ in $u$, i.e.~a choice of a length $2k +1$ element in $\phi^{-1}(s_i : u) \cap R$.

Let $\C_n$ denote $\C \setminus \Set{z_1,\ldots,z_n}$.
Our free group $F$ is $\pi_1(\C_n, z_0)$ for any  $z_0 \notin \Set{z_1,\ldots,z_n} $.
We make the choice of $\Set{z_1,\ldots,z_n}$ to be $n$ evenly spaced points on the imaginary axis which are within the unit circle, and $z_0 = -1$.
We draw a line $\lambda_i$ from each  $z_i$ to $z_0$.
For reasons that will become clear, we draw these lines in a roundabout way, but then for diagramatic readability, we draw that as on the left diagram below.
\[
	\includetikz[baseline=(text57948.center)]{figs/points_on_C_simplified.tex}
	\hspace{-1em}\text{equivalent to}\quad
	\includetikz[baseline=(text57948.center)]{figs/points_on_C.tex}
	.\]
In the following, let each $\epsilon_i$ be in $\Set{\pm 1}$.
A loop $l$ in  $\pi_1(\C_n, z_0)$ represents the element $v = f^{\epsilon_1}_{\alpha_1}\cdots f^{\epsilon_k}_{\alpha_k} \in F$, if $l$ passes through each  $\lambda_i$ in the order specified in  $u$, passing from bottom to top if  $\epsilon_i=1$ and top to bottom if  $\epsilon_i=-1$.
We say that some $v \in F$ is non-crossing if there exists a non-crossing loop which represents $v$.
One can see pictorially that any non-crossing $u = f^{\epsilon_1}_{\alpha_1}\cdots f^{\epsilon_k}_{\alpha_k} \in F$ must be \emph{square free}.
So all $f_{\alpha_i} \neq f_{\alpha_{i+1}}$ and all  $\epsilon_k \in \Set{\pm 1} $ in this setting.

We will want to talk about specific paths in $\C_n$, not just loops up to homotopy.
For this, we use the following definition,
\begin{definition}
	\label{def:free_path}
	Let $\Omega \notin \Set{z_0,\ldots,z_n}$ denote an \emph{end symbol}.
	Let $v$ be some element in $F$.
	A path representing $v\Omega$ is a path (not a loop) that starts at  $z_0$, crosses each $\lambda_i$ in the correct order and orientation as above, but then does not continue back to  $z_0$.
	Once the path has crossed the final $\lambda_k$ (in the correct orientation), it stops a short distance away from that crossing.
\end{definition}
We say that $v\epsilon$ is non-crossing if there exists a non-crossing path that represents  $u\Omega$.
For example, the following is a non-crossing path that represents $f_3^{-1}f_2f_3\Omega$.
\[
	\includetikz[]{figs/non_crossing_path.tex}
\]
A path (possibly a path or a loop) $p$ in $\C_n$ intersects the  $\lambda_i$ at certain points.
We can order these points by closeness to  $z_n$.
We say that a crossing of $p$ across  $\lambda_i$ is \emph{close}, if that crossing is the closest to  $z_i$ along all of  $p$.

\begin{lemma}
	\label{lem:reflection_close_turns}
	Let $v \in F$, and let $f_i \in A$, which is the generating set for $F$.
	The following are equivalent.
	\begin{enumerate}
		\item $f_i : v$ is non-crossing.
		\item There is a non-crossing path representing $vf_i\Omega$ such that the final crossing of $\lambda_i$ is close.
	\end{enumerate}
\end{lemma}
\begin{proof}
	$(2) \implies (1)$ is simple.
	To construct a non-crossing path for $f_i : v$, we follow the path given by (2), then turn in a positive orientation around $z_i$, then follow backwards along our original path, keeping it close on our left as we go back to  $z_0$.
	We follow this procedure in the following example.
	\[
		\includetikz[baseline=(current bounding box.center)]{figs/non_crossing_path_example_start.tex}
		\quad\leadsto\quad
		\includetikz[baseline=(current bounding box.center)]{figs/non_crossing_path_example_end.tex}
	\]
	% In order to show $(1) \implies (2)$, assume we have a non-crossing representative loop $l$ for $f_i : v$, and that the important crossing of $\lambda_i$ is not close.
	We now show $(1) \implies (2)$.
	Let $v = f^{\epsilon_1}_{\alpha_1}f^{\epsilon_2}_{\alpha_2}\cdots f^{\epsilon_n}_{\alpha_k}$.
	Let $l$ be a non-crossing loop representing  $f_i : v$.
	We work from the middle outwards to construct a picture of what $l$ could look like.

	Before turning around $z_i$,  $l$ passed through $\lambda_{\alpha_k}$ and right after turning around  $z_i$, it passed through  $\lambda_{\alpha_k}$ in the opposite direction.
	Similarly, it passed through  $\lambda_{\alpha_{k-1}}$ just before passing through  $\lambda_{\alpha_k}$, and just after in the opposite direction, and so on.
	This gives us the following picture, where $l$ is the submarine shaped line in black.
	\[
		\includetikz[baseline=(current bounding box.center)]{figs/central_turn.tex}
	\]
	This picture is misleading in many ways.
	For instance, there is no reason to assume that $\lambda_x \neq \lambda_y$ for $x \neq y$.
	However, the important features of this picture are the regions, denoted $r_0, \ldots, r_k$ which are bounded by $l$ and the $\lambda$ lines.
	By the above argument, these regions exist exactly as they are depicted in the above picture, in that there are no features inside these regions, apart from $z_i$ inside  $r_k$.

	Now assume there is a part of $l$ that crosses  $\lambda_i$ closer to  $z_i$.
	Being non crossing, this part of $l$ is square free, so must exit (on both ends) the region $r_k$ through $\lambda_{\alpha_k}$.
	Then both ends of this part of  $l$ are in  $r_{k-1}$, and by the same argument both ends must exit  $r_{k-1}$ through  $\lambda_{\alpha_{k-1}}$, and so on, leading to a contradiction.
\end{proof}

\begin{theorem}
	Let $u \in U$ and $s_i \in S$.
	If there is a non-crossing lift $f_i : v$ for $s_i : u$, then it is unique.
\end{theorem}
\begin{proof}
	We do this by inducting on the length of $u$, which we denote $\ell(u)$.
	The statement is vacuous when $\ell(u) = 0$.
	Assume this is true for all $s_i$ and $\ell(u_0) < k$.
	Now suppose we have some reflection $s_i : u$ with $u = u_1\cdots u_{k+1}$ where each $u_i \in S$.
	And suppose this reflection has a non-crossing lift.
	There exists a maximal strict prefix $u^\prime = u_1u_2\cdots u_j$ of  $u$ such that  $u_j : u_1\cdots u_{j-1}$ has a non-crossing lift.
	By induction and by \cref{lem:reflection_close_turns}, there is a unique non-crossing lift for $u_1\cdots u_j \Omega$.
	Thus, to complete our proof, we need to show that the choices of travelling positively or negatively around $u_{j},\ldots,u_k$ are unique.
	We have a choice of how to travel around
\end{proof}


